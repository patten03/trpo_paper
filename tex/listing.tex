\begin{landscape}

\pagestyle{fancy}
\renewcommand{\headrulewidth}{0pt}
\setlength{\headheight}{17pt}
\fancyhf{}
\fancyhead[R]{\rotatebox{90}{\thepage}}

\label{sec:listing}
\section{Листинг программной разработки}

\lstset{
    basicstyle=\ttfamily\fontsize{10pt}{12pt}\selectfont,
    keepspaces=true,
    tabsize=4,
    breaklines=true,
    breakatwhitespace=true
}

\captionsetup[lstlisting]{labelformat=empty}


\begin{lstlisting}[caption={mathVector.h}]

#ifndef MATHVECTOR_H
#define MATHVECTOR_H

#include <cmath>

class MathVector
{
public:
    // Координаты вектора
    float x, y, z;

    // Математические операции для работы с векторами
    // Операции, которые возвращают векторные значения
    MathVector operator+(MathVector const& obj);   // операция сложения
    MathVector operator-(MathVector const& obj);   // операция вычитания
    MathVector crossProd(MathVector const& obj);   // операция векторного умножения
    // Операции, которые возвращают скалярные значения
    float dotProd(MathVector const& obj) const;    // операция скалярного умножения
    float len() const;                             //
    float cosBetween(MathVector const& obj) const; // операция определения косинуса между векторами
    float angle(MathVector const& obj) const;      // операция определения угла между векторами

    // Конструкторы
    MathVector(float xParam = 0, float yParam = 0, float zParam = 0);
    MathVector(MathVector &copy);
};

#endif // MATHVECTOR_H

\end{lstlisting}

\newpage
\begin{lstlisting}[caption={progRes.h}]

#ifndef PROGRES_H
#define PROGRES_H

#include <mathVector.h>

// Структура для хранения вводимых данных пользователем и результатов вычислений
struct Calculations
{
    MathVector A, B; // Векторы A и B
    MathVector       // Векторы суммы, разности и векторного произведения
        Sum,
        Diff,
        CrossProd;
    float
        cosines,     // значения косинуса
        angle,       // значение угла между векторами
        sumLength,   // длина вектора суммы
        diffLength,  // длина вектора разности
        dotProd;     // значение скалярного произведения

    Calculations();
};

#endif // PROGRES_H

\end{lstlisting}

\newpage
\begin{lstlisting}[caption={mainwindow.h}]

#ifndef MAINWINDOW_H
#define MAINWINDOW_H

#include <QMainWindow>
#include <QLineEdit>
#include <QDoubleValidator>

QT_BEGIN_NAMESPACE
namespace Ui {
class MainWindow;
}
QT_END_NAMESPACE

class MainWindow : public QMainWindow
{
    Q_OBJECT

public:
    MainWindow(QWidget *parent = nullptr);
    ~MainWindow();

private slots:
    void on_calculate_clicked();

private:
    Ui::MainWindow *ui;
};
#endif // MAINWINDOW_H

\end{lstlisting}

\newpage
\begin{lstlisting}[caption={main.cpp}]

#include "mainwindow.h"

#include <QApplication>

int main(int argc, char *argv[])
{
    QApplication a(argc, argv);
    MainWindow w;
    w.show();
    return a.exec();
}

\end{lstlisting}

\newpage
\begin{lstlisting}[caption={mathVector.cpp}]

#include "mathVector.h"


// Операция сложения
MathVector MathVector::operator+(MathVector const& obj)
{
    MathVector res;
    res.x = this->x + obj.x;
    res.y = this->y + obj.y;
    res.z = this->z + obj.z;

    return res;
}

// Операция вычитания
MathVector MathVector::operator-(MathVector const& obj)
{
    MathVector res;
    res.x = this->x - obj.x;
    res.y = this->y - obj.y;
    res.z = this->z - obj.z;

    return res;
}

// Операция векторное умножение
MathVector MathVector::crossProd(MathVector const& obj)
{
    MathVector res;
    res.x = this->y * obj.z - this->z * obj.y;
    res.y = this->z * obj.x - this->x * obj.z;
    res.z = this->x * obj.z - this->z * obj.x;

    return res;
}

// Операция скалярного умножения
float MathVector::dotProd(MathVector const& obj) const
{
    float res(0.0);
    res += this->x * obj.x;
    res += this->y * obj.y;
    res += this->z * obj.z;

    return res;
}

// Операция определения длины вектора
float MathVector::len()
const
{
    float res(0.0);
    res += this->x * this->x;
    res += this->y * this->y;
    res += this->z * this->z;

    return sqrt(res);
}

// Операция определения косинуса между векторами
float MathVector::cosBetween(MathVector const& obj) const
{
    return this->dotProd(obj) / this->len() / obj.len();
}

// Операция определения угла между векторами
float MathVector::angle(MathVector const& obj) const
{
    return acos(this->cosBetween(obj));
}

// Конструктор по умолчанию
MathVector::MathVector(float xParam, float yParam, float zParam)
{
    this->x = xParam;
    this->y = yParam;
    this->z = zParam;
}

// Конструктор копирования
MathVector::MathVector(MathVector &copy)
{
    this->x = copy.x;
    this->y = copy.y;
    this->z = copy.z;
}

\end{lstlisting}

\newpage
\begin{lstlisting}[caption={mainwindow.cpp}]

#include "mainwindow.h"
#include "ui_mainwindow.h"
#include "progRes.h"

// Ограничение на количество знаков после запятой
const unsigned short prec = 3;

MainWindow::MainWindow(QWidget *parent)
    : QMainWindow(parent)
    , ui(new Ui::MainWindow)
{
    ui->setupUi(this);
    // Установка заголовка-названия программы
    this->setWindowTitle("Vector Calculator");

    // Установка ограничения на ввод только значений с плавующей точкой от -999999.999 до 999999.999
    QRegularExpression regex("^(-)?([0-9]{1,4})(\\.[0-9]{1," + QString::number(prec) + "})?$");
    QRegularExpressionValidator *validator = new QRegularExpressionValidator(regex, this);
    // Установка ограничений для координат x y z для векторов A и B
    ui->a_x->setValidator(validator);
    ui->a_y->setValidator(validator);
    ui->a_z->setValidator(validator);
    ui->b_x->setValidator(validator);
    ui->b_y->setValidator(validator);
    ui->b_z->setValidator(validator);
}

MainWindow::~MainWindow()
{
    delete ui;
}

// Вычисление данных по нажатию клавиши
void MainWindow::on_calculate_clicked()
{
    // Структура для хранения результатов вычислений
    Calculations Calc;
    // Считывание данных от пользователя
    Calc.A.x = ui->a_x->text().toDouble();
    Calc.A.y = ui->a_y->text().toDouble();
    Calc.A.z = ui->a_z->text().toDouble();
    Calc.B.x = ui->b_x->text().toDouble();
    Calc.B.y = ui->b_y->text().toDouble();
    Calc.B.z = ui->b_z->text().toDouble();

    // Расчет данных
    Calc.Sum = Calc.A + Calc.B;
    Calc.Diff = Calc.A - Calc.B;
    Calc.CrossProd = Calc.A.crossProd(Calc.B);
    Calc.dotProd = Calc.A.dotProd(Calc.B);
    Calc.sumLength = Calc.Sum.len();
    Calc.diffLength = Calc.Diff.len();
    Calc.cosines = Calc.A.cosBetween(Calc.B);
    Calc.angle = Calc.A.angle(Calc.B);

    // Вывод данных на экран
    ui->sum_x->setText(QString::number(Calc.Sum.x, 10, prec));
    ui->sum_y->setText(QString::number(Calc.Sum.y, 10, prec));
    ui->sum_z->setText(QString::number(Calc.Sum.z, 10, prec));

    ui->diff_x->setText(QString::number(Calc.Diff.x, 10, prec));
    ui->diff_y->setText(QString::number(Calc.Diff.y, 10, prec));
    ui->diff_z->setText(QString::number(Calc.Diff.z, 10, prec));

    ui->mul_x->setText(QString::number(Calc.CrossProd.x, 10, prec));
    ui->mul_y->setText(QString::number(Calc.CrossProd.y, 10, prec));
    ui->mul_z->setText(QString::number(Calc.CrossProd.z, 10, prec));

    ui->sumLen->setText(QString::number(Calc.sumLength, 10, prec));
    ui->diffLen->setText(QString::number(Calc.diffLength, 10, prec));
    ui->dotProd->setText(QString::number(Calc.dotProd, 10, prec));

    // Вывод угла между векторами, если значения не существует
    if (std::isnan(Calc.cosines) || std::isnan(Calc.angle))
    {
        ui->cosAngle->setText("Не существует");
        ui->angle->setText("Не существует");
    }
    else
    {
        ui->cosAngle->setText(QString::number(Calc.cosines, 10, prec));
        ui->angle->setText(QString::number(Calc.angle * 180 / 3.14159, 10, prec) + "°");
    }
}

\end{lstlisting}

\end{landscape}