\label{sec:pseudocode}
\section{Псевдокод}

\fontsize{8pt}{10pt}\selectfont
\begin{verbatim}

класс Вектор
{
    x, y, z

    операцияСложения(другой_вектор)
    операцияВычитания(другой_вектор)
    скалярноеУмножение(другой_вектор)
    векторноеУмножение(другой_вектор)
    длина()
    косинусМежду(другой_вектор)
    уголМежду(другой_вектор)
}

Вектор::операцияСложения(другой_вектор)
{
    новый = Вектор()
    новый.x = этот.x + другой_вектор.x
    новый.y = этот.y + другой_вектор.y
    новый.z = этот.z + другой_вектор.z
    вернуть новый
}

Вектор::операцияВычитания(другой_вектор)
{
    новый = Вектор()
    новый.x = этот.x - другой_вектор.x
    новый.y = этот.y - другой_вектор.y
    новый.z = этот.z - другой_вектор.z
    вернуть новый
}

Вектор::векторноеУмножение(другой_вектор)
{
    новый = Вектор()
    новый.x = этот.y * другой_вектор.z - этот.z * другой_вектор.y
    новый.y = этот.z * другой_вектор.x - этот.x * другой_вектор.z
    новый.z = этот.x * другой_вектор.y - этот.y * другой_вектор.x
    вернуть новый
}

Вектор::скалярноеУмножение(другой_вектор)
{
    результат = 0
    результат += этот.x * другой_вектор.x
    результат += этот.y * другой_вектор.y
    результат += этот.z * другой_вектор.z
    вернуть результат
}

Вектор::длина()
{
    результат = 0
    результат += этот.x * этот.x
    результат += этот.y * этот.y
    результат += этот.z * этот.z
    вернуть кореньИз(результат)
}

Вектор::косинусМежду(другой_вектор)
{
    вернуть этот.скалярноеУмножение(другой_вектор) / этот.длина() / другой_вектор.длина()
}

Вектор::уголМежду(другой_вектор)
{
    вернуть арккосинус(этот.косинусМежду(другой_вектор))
}

\end{verbatim}