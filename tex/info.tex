\label{sec:info}
\section{Сведения о программной реализации}

\label{subsec:coding_env}
\subsection{Язык программирования и среда разработки}

При разработке применялись:
\begin{itemize}
    \item Компилятор MinGW 8.1.0
    \item Компоновщик Make 4.4.1
    \item Среда разработки Qt Creator 5.15.0
    \item Стандарт языка ISO C++ 17 Standard
\end{itemize}


\label{subsec:input_output}
\subsection{Входные и выходные данные}

Входные данные:
\begin{itemize}
    \item координаты \textit{(x,y,z)} векторов \textbf{a} и \textbf{b} в вещественном виде.
\end{itemize}

Выходные данные:
\begin{itemize}
    \item координаты \textit{(x,y,z)} вектора суммы \textbf{a+b};
    \item координаты \textit{(x,y,z)} вектора разности \textbf{a-b};
    \item координаты \textit{(x,y,z)} вектора векторного произведения \textbf{a$\times$b};
    \item косинус угла между между векторами \textbf{a} и \textbf{b};
    \item угол между векторами \textbf{a} и \textbf{b};
    \item длина вектора суммы \textbf{|a+b|};
    \item длина вектора разности \textbf{|a-b|};
    \item скалярное произведение векторов \textbf{a} и \textbf{b}.
\end{itemize}


\label{subsec:list_of_modules}
\subsection{Список спроектированных программных единиц}

{
    \setlength{\parindent}{0pt}
    \begin{table}[h!]
        \begin{tabularx}{\textwidth}{ |X|X|X| }
            
        \hline
        Класс&
        Назначение &
        Поля \\ \hline

        \multicolumn{3}{|c|}{Заголовочный файл mathVector.h} \\ \hline

        \code{class MathVector} &
        Предназначен для хранения координат вершины вектора и выполнение операций над вектором. &
        \code{float x, y, z}  - координаты вершины вектора. \\ \hline

        \end{tabularx}
    \end{table}
}

\newpage

{
    \setlength{\parindent}{0pt}
    \begin{table}[h!]
        \begin{tabularx}{\textwidth}{ |>{\raggedright\arraybackslash}X|>{\raggedright\arraybackslash}X|>{\raggedright\arraybackslash}X| }
            
        \hline
        Структура &
        Назначение &
        Поля \\ \hline

        \multicolumn{3}{|c|}{Заголовочный файл progRes.h} \\ \hline

        \code{struct Calculations} &
        Предназначена для хранение результатов операций между векторами. &
        \code{MathVector A, B} - вводимые векторы A и B\newline
        \code{MathVector Sum, Diff, CrossProd} - векторы суммы, разности и векторного произведения\newline
        \code{float cosines} - значения косинуса\newline
        \code{float angle} - значение угла между векторами\newline
        \code{float sumLength} - длина вектора суммы\newline
        \code{float diffLength} - длина вектора разности\newline
        \code{float dotProd} - значение скалярного произведения \\ \hline

        \end{tabularx}
    \end{table}
}

\newpage

\begin{landscape}
    \setlength{\parindent}{0pt}
    
    \begin{longtable}{ |>{\raggedright\arraybackslash}p{6cm}|>{\raggedright\arraybackslash}p{6cm}|>{\raggedright\arraybackslash}p{5cm}|>{\raggedright\arraybackslash}p{5cm}| }
    
    \hline
    Функция &
    Назначение &
    Возвращаемое/изменяемое значение &
    Параметры \\ \hline
    \endhead
    
    \hline
    \endfoot
    
    \multicolumn{4}{|c|}{Заголовочный файл mathVector.h} \\ \hline

    \code{\seqsplit{MathVector\newline MathVector::operator+(MathVector const\& obj)}} &
    Сложения векторов по координатам &
    Возвращает вектор \code{[MathVector]} суммы векторов &
    \code{MathVector\ const\&\ obj} - вектор для сложения \\ \hline
    
    \code{\seqsplit{MathVector\newline MathVector::operator-(MathVector\ const\&\ obj)}} &
    Вычитание векторов по координатам &
    Возвращает вектор \code{[MathVector]} разности векторов &
    \code{MathVector\ const\&\ obj} - вектор для вычитания \\ \hline

    \code{\seqsplit{MathVector\newline MathVector::crossProd(MathVector\ const\&\ obj)}} &
    Векторное умножение векторов &
    Возвращает вектор \code{[MathVector]} векторного произведения векторов &
    \code{MathVector\ const\&\ obj} - вектор для умножения. \\ \hline

    \code{\seqsplit{float\newline MathVector::dotProd(MathVector\ const\&\ obj)\ const}} &
    Вычисление скалярного произведения векторов &
    Возвращает значение скалярного произведения \code{[float]}. &
    \code{MathVector\ const\&\ obj} - вектор для произведения. \\ \hline

    \code{\seqsplit{float\ MathVector::len()}} &
    Определения длины вектора &
    Значение длины вектора \code{[float]}. &
    - \\ \hline

    \code{\seqsplit{float\newline MathVector::cosBetween(MathVector\ const\&\ obj)\ const}} &
    Определения косинуса между векторами. &
    Возвращает значение косинуса \code{[float]} между векторами. &
    \code{MathVector\ const\&\ obj} - вектор, между которым определяется косинус угла. \\ \hline

    \code{\seqsplit{float\newline MathVector::angle(MathVector\ const\&\ obj)\ const}} &
    Определения угла между векторами. &
    Возвращает значение угла  \code{[float]} в радианах между векторами. &
    \code{MathVector\ const\&\ obj} - вектор, между которым определяется угол. \\ \hline

    \code{\seqsplit{MathVector::MathVector(float\ xParam,\ float\ yParam,\ float\ zParam)}} &
    Конструктор по умолчанию. &
    - &
    \code{float xParam, float yParam, float zParam} - координаты вершины вектора (x,y,z) для инициализации вектора. \\ \hline

    \code{\seqsplit{MathVector::MathVector(MathVector \&copy)}} &
    Конструктор копирования. &
    - &
    \code{MathVector \&copy} - вектор для копирования. \\ \hline

    \multicolumn{4}{|c|}{Заголовочный файл mainwindow.h} \\ \hline

    \code{\seqsplit{void\ MainWindow::on\_calculate\_clicked()}} &
    Вычисление данных по нажатию клавиши. &
    - &
    - \\ \hline

    \end{longtable}
\end{landscape}